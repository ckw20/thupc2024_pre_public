\documentclass{article}
\usepackage[UTF8]{ctex}
\usepackage{graphicx}
   \title{B. 一棵树 题解}                   %———总标题
   \author{fzw}
\begin{document}
   \maketitle                                  % —— 显示标题
\tableofcontents                               %—— 制作目录(目录是根据标题自动生成的)

\newpage

\section{B. 一棵树}                             %——一号子标题  China is in East Asia.

注意到当前询问形如权值为边权的权值求和,考虑树形 dp,直接的 dp 为 $f_{i,j}$ 表示 $i$ 子树内有 $j$ 个黑色节点的最小代价。

有转移是易于计算的,子树的合并是树形背包的合并 $f_{x,j+k}=\min(f_{x,j}+f_{to,k})$,考虑当前点 x 到父亲这条边的代价是多少。

注意到,形如如果当前子树有 $A$ 个黑点,子树外有 $k-A$ 个白点,代价为 $|2A-k|$。

观察函数的形式,你注意到如果将 $x$ 坐标作为黑点个数,$y$ 坐标作为代价,则每次代价增量为一个下凸函数,有树形背包合并不影响函数凸性。

注意到凸函数的维护可以尝试维护其差分数组,两个下凸函数做 $\min$ 卷积可以之际的视作差分数组的归并。

现在需要考虑的问题是往父亲方向的增量怎么处理。

考虑代价增量的差分,形如一段前缀差分为负数,一段后缀差分为正数,中间一个点根据 k 的奇偶性讨论是否存在差分 =0。

有这一段前缀的长度总为定长度 $\lfloor \frac k 2 \rfloor$ 于是我们维护可并堆顶堆,使得左侧堆大小始终为前缀定长度,然后可以打加法标记实现。

总复杂度 $n\log n$。



\end{document}
