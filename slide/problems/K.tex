\frame
{
  \frametitle{K 三步棋 {by \itshape SpiritualKhorosho}}

  	给定一个在 $5\times 5$ 的棋盘上进行的简单双人零和完全信息有限博弈.

	初始状态是否为先手必胜取决于给出的 "由不超过 $4$ 枚棋子组成的四连通图案".

	$T$ 组数据,每次给出一个图案,询问是否为先手必胜.
}

\begin{frame}{分析}

	棋盘只有 $5\times 5$,可以考虑记忆化搜索。对任意给定局面,枚举当前操作的玩家将棋子摆在哪个位置,递归求解.

	然而直接记忆化搜索,计算量大概是 $2^{5^2}\times 5^2$ 级别的,再加上多组数据显然无法直接通过.

\end{frame}

\begin{frame}{我会打表!}

	注意到本题保证图案最多包含 $4$ 枚棋子,且本题的规则具有很强的对称性,本质不同的输入只有如下几种:

	\begin{itemize}
	\item 恰有 $1$ 枚棋子,仅 $1$ 种可能性;
	\item 恰有 $2$ 枚棋子,同样只有 $1\times 2$ 这 $1$ 种可能性;
	\item 恰有 $3$ 枚棋子,分为 $1\times 3$ 和 L 形两种;
	\item 恰有 $4$ 枚棋子,可以参考俄罗斯方块,合并旋转对称的 $2$ 对后只有 $5$ 种.
	\end{itemize}

	恰有 $1$ 枚棋子的情况显然;样例中给出了 $1\times 2$ 和 $1\times 3$ 的两种情况的答案,故只需要再打表剩余的 $6$ 种即可.\pause

	精心挑选了一种不能直接根据棋子数量输出答案,但又不会太复杂的规则,希望大家能喜欢.

\end{frame}